Has broadband become a necessity good immune to price changes? Using a 15-year panel of 33 European countries (2010--2024) and two-way fixed effects with Driscoll--Kraay standard errors, we document a fundamental transformation in broadband demand. Pre-COVID, Eastern Partnership countries exhibited highly elastic demand ($\varepsilon = -0.61$, p$<$0.001)---a 10\% price reduction increased subscriptions by 6\%---while EU countries showed moderate elasticity ($\varepsilon = -0.12$, p$<$0.05). By 2020--2024, both regions converged to near-zero elasticity, with price changes having no detectable effect on adoption. Crucially, placebo tests reveal this transformation began in 2015, not 2020, indicating a decade-long digital integration process rather than a COVID-19 shock. We further demonstrate that price measurement critically affects inference: income-relative prices (as \% of GNI) yield significant results in 100\% of specifications, compared to only 25\% for PPP-adjusted prices. These findings have immediate policy relevance: as broadband transitions from discretionary service to essential utility, policy emphasis must shift from affordability subsidies to universal infrastructure deployment.
