This paper documents a fundamental transformation in broadband demand across 33 European countries over 2010--2024. Three findings emerge from our two-way fixed effects analysis with Driscoll--Kraay standard errors.

First, substantial regional heterogeneity characterized pre-COVID demand. Eastern Partnership countries exhibited strong price sensitivity ($\varepsilon = -0.61$, p$<$0.001)---five times larger than EU elasticity ($\varepsilon = -0.12$, p$<$0.05)---reflecting income-driven affordability constraints. Results are robust across 24 specifications.

Second, both regions converged to near-zero elasticity during 2020--2024, with price changes having no detectable effect on adoption. This shift indicates broadband's transformation from discretionary service to essential necessity.

Third, and most critically, this transformation predates COVID-19. Year-by-year estimates reveal declining elasticity from 2015 onward, and placebo tests confirm significant pre-trends (p$=$0.045). The pandemic accelerated but did not initiate broadband's integration into economic and social life.

These findings yield four contributions: updated elasticity estimates for 2010--2024; evidence of regional heterogeneity within Europe; demonstration that demand elasticity is time-varying; and reframing of COVID-19's impact as acceleration rather than shock. Methodologically, we show price measurement critically affects inference---income-relative prices yield robust results while PPP-adjusted prices do not.

For policy, the implications are direct. When demand was elastic (pre-2015), affordability interventions---subsidies, price regulation---effectively expanded adoption. As elasticity approached zero, these tools became ineffective. Policy must now prioritize infrastructure deployment and universal service obligations over price-based mechanisms. The near-zero elasticity suggests broadband resembles utilities like water or electricity, potentially justifying stronger regulatory oversight and public investment to ensure equitable access.

Future research should examine household-level heterogeneity to identify remaining price-sensitive populations, extend analysis to mobile broadband and other regions, and develop quality-adjusted price indices. As broadband completes its transformation into essential infrastructure, understanding these dynamics becomes crucial for bridging digital divides worldwide.
