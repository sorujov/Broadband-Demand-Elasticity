When does a technology transition from luxury to necessity? For broadband internet, this question has profound implications: if demand becomes price-inelastic, subsidies and price regulations---the dominant policy tools of the past two decades---lose their effectiveness. The transformation of broadband from discretionary communication service to essential infrastructure for economic participation, education, healthcare, and civic engagement \citep{katz2010impact, bertschek2016drivers} suggests this transition may already be underway. Yet empirical evidence on whether and when broadband demand became price-insensitive remains surprisingly scarce, despite substantial public investments in broadband infrastructure worldwide \citep{oecd2020broadband}.

This paper provides such evidence by examining how broadband price elasticity evolved over 2010--2024 in 33 countries: 27 European Union (EU) member states and 6 Eastern Partnership (EaP) countries. We leverage substantial cross-country and temporal variation in broadband prices and adoption rates to estimate time-varying elasticities using two-way fixed effects models. Our analysis reveals a striking structural transformation: Eastern Partnership countries exhibited highly price-elastic demand ($\varepsilon = -0.61$) during 2010--2019, but transitioned to price-inelastic demand ($\varepsilon \approx 0$) by 2020--2024, indicating broadband's evolution from discretionary service to essential necessity.

\subsection{Research Questions and Motivation}

Three core questions motivate this research. First, \textit{how has broadband demand elasticity evolved over the past decade and a half?} Existing literature predominantly estimates elasticities at single points in time \citep{grzybowski2015fixed, madden2015demand}, yet technological change and digital economy expansion suggest elasticity may be time-varying as services transition from luxury to necessity \citep{hausman2001price}. 

Second, \textit{what explains regional heterogeneity in price sensitivity?} Eastern Partnership countries---comprising Armenia, Azerbaijan, Belarus, Georgia, Moldova, and Ukraine---represent lower-income markets where affordability concerns might generate higher price elasticity compared to wealthier EU markets. Understanding this heterogeneity is crucial for tailoring policy interventions to different development contexts \citep{waverman2001telecommunications}.

Third, \textit{did COVID-19 fundamentally alter broadband demand?} The pandemic forced abrupt transitions to remote work, online education, and digital service delivery \citep{oecd2021covid}, potentially transforming broadband from convenience to necessity overnight. However, distinguishing COVID's causal impact from pre-existing trends requires careful econometric identification, which we address through placebo tests.

\subsection{Contributions}

This paper makes four principal contributions to telecommunications economics and policy.

\textbf{First}, we document time-varying demand elasticity across a 15-year panel, revealing broadband's structural transformation. Unlike prior single-period estimates, our year-by-year analysis shows elasticity declined gradually from 2015 onward, not suddenly in 2020. This finding challenges the narrative of COVID-19 as an exogenous shock that eliminated price sensitivity, instead suggesting a decade-long evolution driven by digital economy expansion.

\textbf{Second}, we demonstrate that price measurement critically affects empirical inference. Using three alternative price measures---price as percentage of GNI per capita (income-relative), PPP-adjusted prices, and nominal USD prices---we show only income-relative prices yield consistent, statistically significant elasticity estimates. PPP-adjusted prices, commonly used in cross-country comparisons, produce estimates significant in just 25\% of specifications. This methodological contribution has implications beyond broadband demand, highlighting the importance of price measurement in technology goods research where standard PPP baskets may not apply \citep{schreyer2002oecd}.

\textbf{Third}, we provide robust evidence of regional heterogeneity in price sensitivity. Eastern Partnership countries' elasticity ($\varepsilon = -0.61$) was 5.3 times larger than EU elasticity ($\varepsilon = -0.12$) during 2010--2019, consistent with income effects being stronger in developing markets. Importantly, this regional difference diminished during COVID-19, with both regions converging to near-zero elasticity, suggesting broadband became universally essential regardless of income level.

\textbf{Fourth}, we offer policy-relevant insights on the effectiveness of affordability interventions versus infrastructure investment. Our findings suggest telecommunications policy must adapt to broadband's evolving necessity status: price-based interventions (subsidies, universal service obligations) were effective when demand was elastic in the early 2010s, but infrastructure investment and quality standards became more important as demand became inelastic in the 2020s. This has implications for the design of digital inclusion policies worldwide \citep{guermazi2021digital}.

\subsection{Preview of Findings}

Our empirical analysis yields three main findings. First, we estimate baseline pre-COVID (2010--2019) elasticities of $\varepsilon = -0.12$ for EU and $\varepsilon = -0.61$ for EaP using income-relative prices. These estimates are robust across eight alternative control specifications, confirming EaP countries' substantially higher price sensitivity. Second, we find both regions transitioned to near-zero elasticity during 2020--2024, with statistically significant changes ($\Delta\varepsilon = +0.31$ for EU, $\Delta\varepsilon = +0.42$ for EaP). Third, a placebo test splitting the pre-COVID period reveals the EaP-specific trend began in 2015 ($p=0.045$), predating the pandemic and pointing to digital economy expansion as the underlying driver rather than COVID-19 as an exogenous shock.

The remainder of this paper proceeds as follows. Section~\ref{sec:literature} reviews relevant literature on telecommunications demand, digital transformation, and panel data methods. Section~\ref{sec:data} describes our data sources and sample construction. Section~\ref{sec:methodology} presents our empirical strategy. Section~\ref{sec:results} reports baseline and robustness results. Section~\ref{sec:discussion} discusses policy implications and limitations. Section~\ref{sec:conclusion} concludes.
